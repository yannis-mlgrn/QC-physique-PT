
\documentclass{article}
\usepackage[utf8]{inputenc}
\usepackage{amsmath}
\usepackage{mathrsfs}
\usepackage{enumitem} % Pour personnaliser les listes
\usepackage{environ}  % Pour créer des environnements personnalisés
\usepackage{ifthen}   % Pour les conditions
\usepackage[top=2cm, bottom=1.5cm, left=1.5cm, right=1.5cm]{geometry}
% Définir une condition pour afficher ou non les solutions
\newboolean{showsolutions}
\setboolean{showsolutions}{True} % Mettre true pour afficher les solutions

% Définir un environnement pour les solutions
\NewEnviron{solution}{
    \ifthenelse{\boolean{showsolutions}}{
        \noindent\textbf{Solution :} 
        \BODY
    }{}
}

\title{Formulaire et questions de cours en physique-chimie PT}
\author{Yannis}
\date{\today}
\begin{document}
\maketitle
\newpage

\section*{\centering\huge Partie B : Electronique et electromagnétisme}
\section*{\centering Chapitre B4 : Le champ magnétostatique}
\begin{enumerate}[label=\arabic{enumi} - , left=0pt, itemsep=1em] % Personnalisation de la numérotation
    \item Donner le ou les type(s) de sources possible pour un champ magnétostatique. \par
    \begin{solution}
        \begin{itemize}
            \item Aimants
            \item Courrants
        \end{itemize}
    \end{solution}

    \item Définition du courrant.\par
    \begin{solution}
        déplacement de charges électriques.
    \end{solution}

    \item Expression du courrant i en fonction de la charge électrique.\par
    \begin{solution}
        \[ i(t) = \frac{dq(t)}{dt} \]
    \end{solution}


    \item Expression du vecteur densité volumique de courrant en fonction de la densité volumique de courrant.\par
    \begin{solution}
        \[ \vec{j} = \rho \times \vec{v} \]
    \end{solution}

    \item Expression du vecteur densité volumique de courrant en fonction de la charge en P, de la vitesse et de la densité du porteur de charge.\par
    \begin{solution}
         \[ \vec{j} = n q_p \vec{v} \]
    \end{solution}

    \item Expression du vecteur densité volumique de courrant pour plusieurs types de porteurs de charges.\par
    \begin{solution}
         \[ \vec{j} = \sum n_i q_i\vec{v_i} = \sum \rho_i\vec{v_i} \]
    \end{solution}

    \item Unité du vecteur densité volumique de courrant.\par
    \begin{solution}
         \[ A.m^{-2} \]
    \end{solution}

    \item Donner la formule qui relie le courrant i et la densité volumique de courrant.\par
    \begin{solution}
         \[ \int_S \vec{s} \cdot \vec{ds} \]
    \end{solution}

    \item Quelle est la direction du champ magnétique en un point A appartenant à un plan de symétrie de la distribution de courrant  \par
    \begin{solution}
         \[ \vec{B(A)} \perp \pi_S \]
    \end{solution}


    \item Quelle est la propriété fondamentale liée au flux du vecteur du champ magnétostatique?\par 
    \begin{solution}
         \[ \oint_S \vec{B}(M) \cdot \vec{ds} = \vec{0} \]
    \end{solution}

    \item Donner la formule de la circulation de $\vec{B}$ le long du contour fermé orienté $\Gamma$\par
    \begin{solution}
         \[ \mathscr{C}(\vec{B}) = \oint_S \vec{B}(M) \cdot \vec{dOM} \]
    \end{solution}

    \item Donner la définition de $i_{\text{enlacés}}$ \par
    \begin{solution}
         Courrant enlacé par le contour fermé orienté $\Gamma$
    \end{solution}

    \item Énoncer Le théorème d'Ampère \par
    \begin{solution}
         \[ \mathscr{C}(\vec{B}) = \nu_0 \times (i_{\text{enlacés}})\]
    \end{solution}
    
    \item Donner l'expression du champ magnétostatique à l'interieur d'un solénoïde infini.\par
    \begin{solution}
         \[ \vec{B(M)} = \nu_0 n i . \vec{u_z}\] n : nombre de spire du solénoïde
    \end{solution}

    \item Donner l'expression du champ magnétostatique à l'exterieur d'un cylindre infini.\par
    \begin{solution}
         \[ \vec{B(M)} = \frac{\nu_0 I}{2 \pi r} . \vec{u_{\theta}}\] I : courrants enlacés
    \end{solution}

    \item  $\textbf{Démonstration :}$ Etablir l’expression du champ magnétostatique généré en tout point de l’espace par un
    cylindre infini, de rayon R, parcouru par un courant de densité volumique uniforme.\par
    \begin{solution}
         Méthode d'ampère, Voir démo V.2

    \end{solution}


    \item $\textbf{Démonstration :}$ Etablir l’expression du champ magnétostatique généré en tout point de l’espace par un
    solénoïde infini parcouru par un courant i.\par
    \begin{solution}
         Méthode d'ampère, Voir démo V.3

    \end{solution}

\section*{\centering\huge Partie C : Thermodynamique et mécanique des fluides appliquées aux machines thermiques}

\section*{\centering Chapitre C1 : Statique des fluides}
    \item Donner la définition d'une particule de fluide. \par
    \begin{solution}
     Une particule de fluide est un système fermé constitué par la masse $\delta_M$ de fluide de volume mésoscopique $d\tau$
    \end{solution}

    \item Donner la définition de la statique des fluides. \par
    \begin{solution}
     La statique des fluides, c'est l'étude de l'équilibre des particules de fluides 
    \end{solution}

    \item Donner La définition et l'expression de la force surfacique qui s'exerce sur la surface de la particule de fluide en M. \par
    \begin{solution}
     les forces surfaciques sont les forces qui s'exercent sur la surface de la particule de fluide
          \[ \overrightarrow{dF}_{fluide-int \to fluide-ext} = P(M) \vec{dS}\]
         
          \begin{itemize}
               \tiny\item   P : pression en Pa
               \tiny\item   P(M) : champ scalaire positif
           \end{itemize}  
     \end{solution}

     \item $\textbf{Démonstration :}$ Etablir la relation fondamentale de la statique des fluides.\par
           \begin{solution}
               Voir démo II.1, \[ \frac{dP}{dz}=-\rho g \]
       
           \end{solution}
     
     \item Donner la valeur de la masse molaire de l'air\par
           \begin{solution}
               \[ M_{air} = 29 g.mol^{-1} \]
       
           \end{solution}
     
    \item Donner la définition de la poussée d'archimède. \par
    \begin{solution}
        La poussée d'archimède est la résultante des forces de pression qui s'exercent sur toutes les surfaces du solide 
     \end{solution}

     \item Donner l'expression de la poussée d'archimède. \par
     \begin{solution}
         \[ \vec{\pi_A} = -\oint_{M \in S} P(M) \cdot \vec{dS} \]
         \begin{itemize}
          \tiny\item   S : surface de l'objet
          \tiny\item   $\vec{dS}$ : orienté par la normale sortante
      \end{itemize}  
      \end{solution}

      \item Enoncer et donner l'expression du théorème d'archimède. \par
      \begin{solution}
          La poussée d'archimède est égale à l'opposé du poid du fluide déplacé
          \[\vec{\pi_A} = - \overrightarrow{Pf_d} \] 
       \end{solution}

       \item Donner la relation de la statique des fluides imcompressibles. \par
       \begin{solution}
          \[ P + \rho g z = cste\]
        \end{solution}

        \item Donner la relation de la statique des fluides imcompressibles en un point M à une profondeur H d'un fluide au contact d'un autre fluide de presssion $P_o$. \par
        \begin{solution}
           \[P(M) = P_o + \rho g H\]
           
         \end{solution}

         \item En quel point s'applique la poussée d'archimède $\pi_A$ ? \par
         \begin{solution}
            Au centre de poussée C
          \end{solution}


         \item Donner et expliquer les 3 méthodes pour calculer les forces de pressions \par
         \begin{solution}
            voir cours IV : Utilisation du poid, de la poussée d'archimède et par intégration directe
          \end{solution}
\end{enumerate}
\end{document}