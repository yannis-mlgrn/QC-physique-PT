\section*{\centering Chapitre A2 : Rétroaction}
\begin{enumerate}[label=\arabic{enumi} - , left=0pt, itemsep=1em] % Personnalisation de la numérotation
    \item Donner la définition d'un amplificateur idéal de tension. \par
    \begin{solution}
          un amplificateur de tension idéal est un système electronique qui augmente la tension d'un signal electrique. À ce moment, il va définir plus de puissance en sortie qu'en entrée et il va être alimenté. 
    \end{solution}


    \item Donner les caractéristiques d'un amplificateur idéal. \par
    \begin{solution}\\
        $Z_e = \infty$\\
        $Z_s = 0$
    \end{solution}

    \item Donner le schéma d'un ALI idéal de tension. \par
    \begin{solution}\\
        Cf cours I.1
    \end{solution}

    \item Donner la caractéristiques de transfert statique et donner ses valeurs en régime linéaire et saturé. \par
    \begin{solution}\\
        Cf cours I.4 \\
        régime linéaire : s = $\nu_0 \times \epsilon$ \\
        régime saturé : s = $\pm V_{sat}$
    \end{solution}

    \item Donner le gain caractéristique d'un ALI. \par
    \begin{solution}\\
        Gain : $\sim 10^5$
    \end{solution}

    \item Donner l'impédance d'entée d'un ALI. \par
    \begin{solution}\\
        impédance : $> 1 M \Omega$
    \end{solution}

    \item Donner l'impédance de sortie d'un ALI. \par
    \begin{solution}\\
        impédance : $< 0.1 \nu A$
    \end{solution}

    \item Donner les caractéristiques d'un amplificateur idéal en régime linéaire. \par
    \begin{solution}\\
        $Z_e = \infty$\\
        $Z_s = 0$\\
        $i^+ = i^- = 0$
    \end{solution}

    \item Donner le schéma du montage, le schéma fonctionnel et la fonction de transfert d'un amplificateur non inverseur. \par
    \begin{solution}\\
        Cf cours II.2
    \end{solution}


    \item Donner le schéma du montage et la fonction de transfert d'un comparateur à hystérésis inverseur. \par
    \begin{solution}\\
        Cf cours II.3
    \end{solution}

    \item Donner les liens entre le bouclage et le régime de l'ALI. \par
    \begin{solution}\\
        - Bouclage entre $E^-$ et S le montage fonctionne probablement en régime linéaire\\
        - bouclage entre $E^+$ ou si il n'y a pas de bouclage entre $E^-$ et S, le montage fonctionne forcément en régime saturé\\
        - Bouclage entre $E^-$ et S et $E^+$ et S, le bouclage fonctionne probablement en régimé linéaire.

    \end{solution}

    \item Donner la valeur du gain, de $\nu_o$ et de la tension differentielle $\epsilon$ d'un ALI idéal de gain $\infty$ en régime linéaire \par
    \begin{solution}\\
        Gain : $\infty$\\
        $\nu_o$ : $\infty$\\
        $\epsilon$ : 0

    \end{solution}

    \item Donner le schéma du montage et la relation entre e et s d'un amplificateur non inverseur de gain  $\infty$  en régime linéaire. \par
    \begin{solution}\\
        Cf cours III.3
    \end{solution}

    \item Donner le schéma du montage et la relation entre e et s d'un amplificateur inverseur de gain  $\infty$ en régime linéaire. \par
    \begin{solution}\\
        Cf cours III.4
    \end{solution}

    \item  $\textbf{Démonstration :}$ Donner le schéma du montage et les conditions d'un comparateur simple de gain  $\infty$ en régime saturé \par
    \begin{solution}
        Cf cours IV.2
    \end{solution}

    \item  $\textbf{Démonstration :}$ Donner le schéma du montage et les conditions d'un comparateur à hystérésis inverseur de gain  $\infty$ en régime saturé \par
    \begin{solution}
        Cf cours IV.3
    \end{solution}

    \item  $\textbf{Démonstration :}$ Donner le schéma du montage et les conditions d'un comparateur à hystérésis non inverseur de gain  $\infty$ en régime saturé \par
    \begin{solution}
        Cf TD ex5
    \end{solution}
    
\end{enumerate}

