\section*{\centering Chapitre A3 : Oscillateurs}
\begin{enumerate}[label=\arabic{enumi} - , left=0pt, itemsep=1em] % Personnalisation de la numérotation

    \item Donner la définition d'un Oscillateur. \par
    \begin{solution}
          Un Oscillateur est un circuit electrique qui délivre en sortie une tension periodique sans tension d'entrée. Il doit 
          être composé d'un élément actif : un ALI et d'un élément non-linéaire pour bloquer les oscillations
    \end{solution}

    \item Donner la condition de Barkhauseur. \par
    \begin{solution} \\
        AB = 1\\
        A : Bloc d'action \\
        B : bloc de réaction\\
        Cf cours I.1.C
    \end{solution}

    \item Donner les origines de la petite tension initiale qui permet de démarrer l'oscillateur. \par
    \begin{solution} \\
        - Tensions parasites de l'ALI ( soudures, offset de l'ALI, courrants de polarisation)
    \end{solution}
    
\end{enumerate}


