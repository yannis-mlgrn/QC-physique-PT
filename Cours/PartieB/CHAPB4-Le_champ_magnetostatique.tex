\section*{\centering Chapitre B4 : Le champ magnétostatique}
\begin{enumerate}[label=\arabic{enumi} - , left=0pt, itemsep=1em] % Personnalisation de la numérotation
   \item Donner le ou les type(s) de sources possible pour un champ magnétostatique. \par
    \begin{solution}
          Aimants, Courrants
    \end{solution}

    \item Définition du courrant.\par
    \begin{solution}
        déplacement de charges électriques.
    \end{solution}

    \item Expression du courrant i en fonction de la charge électrique.\par
    \begin{solution}
        \[ i(t) = \frac{dq(t)}{dt} \]
    \end{solution}


    \item Expression du vecteur densité volumique de courrant en fonction de la densité volumique de courrant.\par
    \begin{solution}
        \[ \vec{j} = \rho \times \vec{v} \]
    \end{solution}

    \item Expression du vecteur densité volumique de courrant en fonction de la charge en P, de la vitesse et de la densité du porteur de charge.\par
    \begin{solution}
         \[ \vec{j} = n q_p \vec{v} \]
    \end{solution}

    \item Expression du vecteur densité volumique de courrant pour plusieurs types de porteurs de charges.\par
    \begin{solution}
         \[ \vec{j} = \sum n_i q_i\vec{v_i} = \sum \rho_i\vec{v_i} \]
    \end{solution}

    \item Unité du vecteur densité volumique de courrant.\par
    \begin{solution}
         \[ A.m^{-2} \]
    \end{solution}

    \item Donner la formule qui relie le courrant i et la densité volumique de courrant.\par
    \begin{solution}
         \[ \int_S \vec{s} \cdot \vec{ds} \]
    \end{solution}

    \item Quelle est la direction du champ magnétique en un point A appartenant à un plan de symétrie de la distribution de courrant  \par
    \begin{solution}
         \[ \vec{B(A)} \perp \pi_S \]
    \end{solution}


    \item Quelle est la propriété fondamentale liée au flux du vecteur du champ magnétostatique?\par 
    \begin{solution}
         \[ \oint_S \vec{B}(M) \cdot \vec{ds} = \vec{0} \]
    \end{solution}

    \item Donner la formule de la circulation de $\vec{B}$ le long du contour fermé orienté $\Gamma$\par
    \begin{solution}
         \[ \mathscr{C}(\vec{B}) = \oint_S \vec{B}(M) \cdot \vec{dOM} \]
    \end{solution}

    \item Donner la définition de $i_{\text{enlacés}}$ \par
    \begin{solution}
         Courrant enlacé par le contour fermé orienté $\Gamma$
    \end{solution}

    \item Énoncer Le théorème d'Ampère \par
    \begin{solution}
         \[ \mathscr{C}(\vec{B}) = \nu_0 \times (i_{\text{enlacés}})\]
    \end{solution}
    
    \item Donner l'expression du champ magnétostatique à l'interieur d'un solénoïde infini.\par
    \begin{solution}
         \[ \vec{B(M)} = \nu_0 n i . \vec{u_z}\] n : nombre de spire du solénoïde
    \end{solution}

    \item Donner l'expression du champ magnétostatique à l'exterieur d'un cylindre infini.\par
    \begin{solution}
         \[ \vec{B(M)} = \frac{\nu_0 I}{2 \pi r} . \vec{u_{\theta}}\] I : courrants enlacés
    \end{solution}

    \item  $\textbf{Démonstration :}$ Etablir l’expression du champ magnétostatique généré en tout point de l’espace par un
    cylindre infini, de rayon R, parcouru par un courant de densité volumique uniforme.\par
    \begin{solution}
         Méthode d'ampère, Voir démo V.2

    \end{solution}


    \item $\textbf{Démonstration :}$ Etablir l’expression du champ magnétostatique généré en tout point de l’espace par un
    solénoïde infini parcouru par un courant i.\par
    \begin{solution}
         Méthode d'ampère, Voir démo V.3

    \end{solution}

\end{enumerate}