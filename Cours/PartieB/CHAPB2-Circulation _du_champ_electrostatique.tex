\section*{\centering Chapitre B2 : Circulation du champ electrostatique}
\begin{enumerate}[label=\arabic{enumi} - , left=0pt, itemsep=1em] % Personnalisation de la numérotation

    \item Donner l'expression de l'energie potencielle de la charge ponctuelle q en intéraction abec la charge ponctuelle $q_o$. \par
    \begin{solution} \\
        \[ Ep = \frac{1}{4 \pi \epsilon_o} \times \frac{q_o q}{r} + cste\]
    \end{solution}

    \item Donner l'expression de l'energie potencielle de la charge ponctuelle q pour une distribution de charge continue. \par
    \begin{solution} \\
        \[ Ep = \frac{q}{4 \pi \epsilon_o} \times \int_{P \epsilon Distrib} \frac{dq(P)}{PM} + cste\]
    \end{solution}

    \item Donner la définition d'un potenciel electrostatique. \par
    \begin{solution} \\
        Le potenciel V au point M est tel que si on y place une charge q, elle va acquérir une énergie potencielle proportionnelle à q.
        Le point M rend compte de cette propriété en introduisant le scalaire : le potenciel electrostatique V(M)
    \end{solution}


    \item Donner l'expression de Ep en fonction de V(M). \par
    \begin{solution} \\
        Ep = qV(M)
    \end{solution}

    \item Donner l'expression du potenciel electrostatique au point M de la charge ponctuelle $q_o$ située en O.\par
    \begin{solution} \\
        \[ V(M) = \frac{q}{4 \pi \epsilon_o} \times \sum_i \frac{q_i}{P_iM} + cste\]
    \end{solution}

    \item Donner l'expression du potenciel electrostatique au point M de la charge q pour une distribution de charge ponctuelles continue $q_i$ en $P_i$. \par
    \begin{solution} \\
        \[ V(M) = \frac{1}{4 \pi \epsilon_o} \times \frac{q_o}{r} + cste\]
    \end{solution}
    
    \item Donner l'expression du potenciel electrostatique de la charge ponctuelle q pour une distribution de charge continue. \par
    \begin{solution} \\
        \[ Ep = \frac{1}{4 \pi \epsilon_o} \times \int_{P \epsilon Distrib} \frac{dq(P)}{PM} + cste\]
    \end{solution}

    \item Donner le domaine de définition et de continuité du potenciel electrostatique V(M). \par
    \begin{solution} \\
        V(M) est continu et défini en tout point, sauf sur une distribution de charge linéïque et sur une charge ponctuelle.
    \end{solution}

    \item Donner l'expression de la circulation du champ $\vec{a}$ le long d'une courbe orienté $\Gamma$. \par
    \begin{solution} \\
        \[ \mathcal{C} (M) = \int_{M \epsilon \Gamma} \vec{a}(M) \cdot \vec{dOM} \]
    \end{solution}

    \item Donner l'expression de la circulation du champ $\vec{a}$ le long d'un contour fermé $\Gamma$. \par
    \begin{solution} \\
        \[ \mathcal{C} (M) = \oint_{M \epsilon \Gamma} \vec{a}(M) \cdot \vec{dOM} \]
    \end{solution}

    \item  $\textbf{Démonstration :}$ Etablir que $\vec{E} \cdot \vec{dOM} = -dV$.\par
    \begin{solution}
         Voir démo II.b

    \end{solution}

    \item Donner la propriété fondamentale du champ $\vec{E}$. \par
    \begin{solution} \\
        \[\oint_{M \epsilon \Gamma} \vec{E}(M) \cdot \vec{dOM} = 0 \]\\
        On dit que $\vec{E}$ est à circulation conservative
    \end{solution}

    \item Donner la définition d'une surface équipotencielle. \par
    \begin{solution} \\
        c'est une surface où V=cste ( potenciel constant)
    \end{solution}

    \item Donner la définition de ligne de champ. \par
    \begin{solution} \\
        les lignes de champ sont des lignes $\perp$ aux surfaces équipotencielles
    \end{solution}

    \item Donner l'expression du champ E en fonction du gradient. \par
    \begin{solution} \\
        \[ E = - \vec{grad} V \]\\
        "la force E dérive d'un potenciel"
    \end{solution}

    \item Donner l'expression de la force F en fonction du gradient. \par
    \begin{solution} \\
        \[ F = - \vec{grad} Ep \]\\
        "la force F dérive d'une Ep"
    \end{solution}
\end{enumerate}