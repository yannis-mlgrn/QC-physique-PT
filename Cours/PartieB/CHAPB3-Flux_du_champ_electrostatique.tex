\section*{\centering Chapitre B3 : Flux du champ electrostatique}
\begin{enumerate}[label=\arabic{enumi} - , left=0pt, itemsep=1em] % Personnalisation de la numérotation

    \item Donner l'expression du flux d'un champ de vecteur. \par
    \begin{solution} \\
        \[ \phi ( \vec{a}) = \int_{M \epsilon S} \vec{a}(M) \cdot \vec{ds}\]
    \end{solution}

    \item Donner la définition d'une surface fermée. \par
    \begin{solution} \\
        Une surface fermée délimite un volume intérieur d'un volume exterieur 
    \end{solution}

    \item Énoncer le théorème de Gauss. \par
    \begin{solution} \\
        Le théorème de Gauss généralise l'expression du flux pour n'importe quelle charge $Q_{int}$ à l'interieur de n'importe quelle surface fermée S\\
        \[ \phi(\vec{E}) = \oint \vec{E}(M) \cdot \vec{ds} = \frac{Qint}{\epsilon_o}\]
    \end{solution}

    \item $\textbf{Démonstration :}$ Etablir l’expression du champ électrostatique généré en tout point de l’espace par une
    sphère de rayon R portant une densité volumique de charge $\rho$ uniforme. \par
    \begin{solution} \\
        Cf cours III.2
    \end{solution}

    \item $\textbf{Démonstration :}$ Etablir l’expression, en tout point de l’espace, du champ électrostatique généré par un
    plan infini portant une charge surfacique uniforme, de densité $\sigma$. \par
    \begin{solution} \\
        Cf cours III.4
    \end{solution}

    \item $\textbf{Démonstration :}$ Rappeler l’expression du champ électrostatique généré par un plan infini portant une
    charge surfacique uniforme, de densité $\sigma$ ; puis établir l’expression du champ
    électrostatique à l’intérieur d’un condensateur plan, et de la capacité du condensateur. $\sigma$. \par
    \begin{solution} \\
        Cf cours III.4 et III.4.a
    \end{solution}

    \item $\textbf{Démonstration :}$ Etablir l’expression du champ électrostatique généré en tout point de l’espace par un
    cylindre infini, de rayon R, portant une charge volumique uniforme, de densité $\rho $ \par
    \begin{solution} \\
        Cf TD 6.1
    \end{solution}
\end{enumerate}