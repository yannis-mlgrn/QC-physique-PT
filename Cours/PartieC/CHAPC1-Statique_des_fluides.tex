\section*{\centering Chapitre C1 : Statique des fluides}
\begin{enumerate}[label=\arabic{enumi} - , left=0pt, itemsep=1em] % Personnalisation de la numérotation
    \item Donner la définition d'une particule de fluide. \par
    \begin{solution}
     Une particule de fluide est un système fermé constitué par la masse $\delta_M$ de fluide de volume mésoscopique $d\tau$
    \end{solution}

    \item Donner la définition de la statique des fluides. \par
    \begin{solution}
     La statique des fluides, c'est l'étude de l'équilibre des particules de fluides 
    \end{solution}

    \item Donner La définition et l'expression de la force surfacique qui s'exerce sur la surface de la particule de fluide en M. \par
    \begin{solution}
     les forces surfaciques sont les forces qui s'exercent sur la surface de la particule de fluide
          \[ \overrightarrow{dF}_{fluide-int \to fluide-ext} = P(M) \vec{dS}\]
         
          \begin{itemize}
               \tiny\item   P : pression en Pa
               \tiny\item   P(M) : champ scalaire positif
           \end{itemize}  
     \end{solution}

     \item $\textbf{Démonstration :}$ Etablir la relation fondamentale de la statique des fluides.\par
           \begin{solution}
               Voir démo II.1, \[ \frac{dP}{dz}=-\rho g \]
       
           \end{solution}
     
     \item Donner la valeur de la masse molaire de l'air\par
           \begin{solution}
               \[ M_{air} = 29 g.mol^{-1} \]
       
           \end{solution}
     
    \item Donner la définition de la poussée d'archimède. \par
    \begin{solution}
        La poussée d'archimède est la résultante des forces de pression qui s'exercent sur toutes les surfaces du solide 
     \end{solution}

     \item Donner l'expression de la poussée d'archimède. \par
     \begin{solution}
         \[ \vec{\pi_A} = -\oint_{M \in S} P(M) \cdot \vec{dS} \]
         \begin{itemize}
          \tiny\item   S : surface de l'objet
          \tiny\item   $\vec{dS}$ : orienté par la normale sortante
      \end{itemize}  
      \end{solution}

      \item Enoncer et donner l'expression du théorème d'archimède. \par
      \begin{solution}
          La poussée d'archimède est égale à l'opposé du poid du fluide déplacé
          \[\vec{\pi_A} = - \overrightarrow{Pf_d} \] 
       \end{solution}

       \item Donner la relation de la statique des fluides imcompressibles. \par
       \begin{solution}
          \[ P + \rho g z = cste\]
        \end{solution}

        \item Donner la relation de la statique des fluides imcompressibles en un point M à une profondeur H d'un fluide au contact d'un autre fluide de presssion $P_o$. \par
        \begin{solution}
           \[P(M) = P_o + \rho g H\]
           
         \end{solution}

         \item Énoncer le théorème de Pascal. \par
         \begin{solution}
            Les fluides imcompressibles transmettent intégralement les variations de préssion : $\Delta P' = \Delta P $
            
          \end{solution}

         \item En quel point s'applique la poussée d'archimède $\pi_A$ ? \par
         \begin{solution}
            Au centre de poussée C
          \end{solution}


         \item Donner et expliquer les 3 méthodes pour calculer les forces de pressions \par
         \begin{solution}
            voir cours IV : Utilisation du poid, de la poussée d'archimède et par intégration directe
          \end{solution}
\end{enumerate}
\end{document}