
\documentclass{article}
\usepackage[utf8]{inputenc}
\usepackage{amsmath}
\usepackage{mathrsfs}
\usepackage{enumitem} 
\usepackage{environ}  
\usepackage{ifthen}   
\usepackage{subfiles} 
\usepackage[top=2cm, bottom=1.5cm, left=1.5cm, right=1.5cm]{geometry}
\usepackage{tikz}
\newboolean{showsolutions}
\setboolean{showsolutions}{True}

% Définir un environnement pour les solutions
\NewEnviron{solution}{
    \ifthenelse{\boolean{showsolutions}}{
        \noindent\textbf{Solution :} 
        \BODY
    }{}
}
% Page de couverture
\title{Révisions et Questions de cours en physique-chimie PT}
\author{Yannis Malgorn}
\date{\today}
\begin{document}
\maketitle
\newpage

%Affichage des questions de cours
\section*{\centering\huge Partie A : Electronique}
\subfile{../Cours/PartieA/CHAPA1-Stabilite_des_systemes_lineaires.tex}
\subfile{../Cours/PartieA/CHAPA2-Retroaction.tex}

\subfile{../Cours/PartieA/CHAPA3-Oscillateurs.tex}
% Partie B 
\section*{\centering\huge Partie B : Electronique et electromagnétisme} 
\subfile{../Cours/PartieB/CHAPB1-Le_champ_electrostatique}
\subfile{../Cours/PartieB/CHAPB2-Circulation _du_champ_electrostatique.tex}
\subfile{../Cours/PartieB/CHAPB3-Flux_du_champ_electrostatique.tex}
\subfile{../Cours/PartieB/CHAPB4-Le_champ_magnetostatique}
% Partie C
\section*{\centering\huge Partie C : Thermodynamique et mécanique des fluides appliquées aux machines thermiques}
\subfile{../Cours/PartieC/CHAPC1-Statique_des_fluides}
\end{document}