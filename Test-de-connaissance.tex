
\documentclass{article}
\usepackage[utf8]{inputenc}
\usepackage{amsmath}
\usepackage{mathrsfs}
\usepackage{enumitem} 
\usepackage{environ}  
\usepackage{ifthen}   
\usepackage{subfiles} 
\usepackage[top=2cm, bottom=1.5cm, left=1.5cm, right=1.5cm]{geometry}
\newboolean{showsolutions}

% Définir un environnement pour les solutions
\NewEnviron{solution}{
    \ifthenelse{\boolean{showsolutions}}{
        \noindent\textbf{Solution :} 
        \BODY
    }{}
}
\begin{document}
% Affichage des questions de cours
\section*{\centering\huge Chapitre B1 : Le champ electrostatique}
\setboolean{showsolutions}{False}
\begin{enumerate}[label=\arabic{enumi} - , left=0pt, itemsep=1em]
    \item Donner l'expression de la charge en fonction de la densité volumique de charge $\rho$. \par
    \begin{solution}
          \[ dq = \rho \times d\tau \] et \[ q = \int_\tau dq = \int_\tau \rho \times d\tau \]
    \end{solution}
    \item Donner l'expression de la charge lors d'une distribution surfacique de charge $\sigma $. \par
    \begin{solution}
          \[ dq = \sigma \times dS \] et \[ q = \int_S \sigma \times dS \]
          $\sigma$ en $C.m^{-2}$
    \end{solution}
    \item Enoncer l'expression la loi de Coulomb. \par
    \begin{solution}
          \[ \overrightarrow{F_{P \to M}} = \frac{1}{4 \pi \epsilon_o} \times \frac{q_pq}{PM^2} \]
          Accompagné de son schéma II.1

           $\epsilon_o$ : perméabilité diélectrique du vide en $F.m^{-1}$
    \end{solution}
    \item Donner l'expression du champ electrostatique $\vec{E}$. \par
    \begin{solution}
          Le champs $\vec{E}$ au point M est tel que si on y place une charge q, elle serait soumise à une force electrostatique.
    \end{solution}
    \item Donner l'expression de la force electrostatique. \par
    \begin{solution}
          \[ \vec{F} = q \times E(M)\]
    \end{solution}
    \item Donner l'expression du champ electrostatique créé par la charge potencielle $q_p$ en P au point M. \par
    \begin{solution}
          \[ \vec{E(M)} = \frac{1}{4 \pi \epsilon_o} \times \frac{q_p}{PM^2} \cdot \vec{u_{PM}} \]
    \end{solution}
    \item Donner l'expression du champ electrostatique créé par une charge $q_p$ en P au point M grâce à une distribution continue de charge. \par
    \begin{solution}
          \[ \vec{E(M)} = \int_{P \epsilon distrib} d\vec{E(M)} = \frac{1}{4 \pi \epsilon_o} \int_{P \epsilon distrib} \frac{dq(P)}{PM^2} \cdot \vec{u_{PM}} \]
          avec : 
          \[ \int_{P \epsilon distrib} dq(P)  = \int_\tau \rho(P) \times d\tau =\int_S \sigma(P) \times dS = \int_L \lambda(P) \times dl \]   
  
    \end{solution}
    \item Donner le domaine de définition du champ $\vec{E(M)}$ suivant les distributions. \par
    \begin{solution}
     $\vec{E(M)}$ est défini partout pour une distribution volumique mais n'est pas défini sur les autres distributions (surfacique, volumique)
  
    \end{solution}
    \item Donner la valeur du champ d'ionisation de l'air. \par
    \begin{solution}
     36kV.$cm^{-1}$
  
    \end{solution}
        \item Enoncer le principe de Curie. \par
    \begin{solution}
     Les effets sont au moins aussi symétrique que les causes
     \end{solution}
     \item Donner la définition de ligne de champs \par
     \begin{solution}
      ligne orienté tangent au champ à chacun de ses points
     \end{solution}
     \item Donner la définition de tube de champs \par
     \begin{solution}
      Ensemble des lignes de champ qui s'appuient sur un contour fermé 
     \end{solution}
\end{enumerate}
\newpage
\section*{\centering\huge Réponses :}
\setboolean{showsolutions}{True}
\begin{enumerate}[label=\arabic{enumi} - , left=0pt, itemsep=1em]
    \item Donner l'expression de la charge en fonction de la densité volumique de charge $\rho$. \par
    \begin{solution}
          \[ dq = \rho \times d\tau \] et \[ q = \int_\tau dq = \int_\tau \rho \times d\tau \]
    \end{solution}
    \item Donner l'expression de la charge lors d'une distribution surfacique de charge $\sigma $. \par
    \begin{solution}
          \[ dq = \sigma \times dS \] et \[ q = \int_S \sigma \times dS \]
          $\sigma$ en $C.m^{-2}$
    \end{solution}
    \item Enoncer l'expression la loi de Coulomb. \par
    \begin{solution}
          \[ \overrightarrow{F_{P \to M}} = \frac{1}{4 \pi \epsilon_o} \times \frac{q_pq}{PM^2} \]
          Accompagné de son schéma II.1

           $\epsilon_o$ : perméabilité diélectrique du vide en $F.m^{-1}$
    \end{solution}
    \item Donner l'expression du champ electrostatique $\vec{E}$. \par
    \begin{solution}
          Le champs $\vec{E}$ au point M est tel que si on y place une charge q, elle serait soumise à une force electrostatique.
    \end{solution}
    \item Donner l'expression de la force electrostatique. \par
    \begin{solution}
          \[ \vec{F} = q \times E(M)\]
    \end{solution}
    \item Donner l'expression du champ electrostatique créé par la charge potencielle $q_p$ en P au point M. \par
    \begin{solution}
          \[ \vec{E(M)} = \frac{1}{4 \pi \epsilon_o} \times \frac{q_p}{PM^2} \cdot \vec{u_{PM}} \]
    \end{solution}
    \item Donner l'expression du champ electrostatique créé par une charge $q_p$ en P au point M grâce à une distribution continue de charge. \par
    \begin{solution}
          \[ \vec{E(M)} = \int_{P \epsilon distrib} d\vec{E(M)} = \frac{1}{4 \pi \epsilon_o} \int_{P \epsilon distrib} \frac{dq(P)}{PM^2} \cdot \vec{u_{PM}} \]
          avec : 
          \[ \int_{P \epsilon distrib} dq(P)  = \int_\tau \rho(P) \times d\tau =\int_S \sigma(P) \times dS = \int_L \lambda(P) \times dl \]   
  
    \end{solution}
    \item Donner le domaine de définition du champ $\vec{E(M)}$ suivant les distributions. \par
    \begin{solution}
     $\vec{E(M)}$ est défini partout pour une distribution volumique mais n'est pas défini sur les autres distributions (surfacique, volumique)
  
    \end{solution}
    \item Donner la valeur du champ d'ionisation de l'air. \par
    \begin{solution}
     36kV.$cm^{-1}$
  
    \end{solution}
        \item Enoncer le principe de Curie. \par
    \begin{solution}
     Les effets sont au moins aussi symétrique que les causes
     \end{solution}
     \item Donner la définition de ligne de champs \par
     \begin{solution}
      ligne orienté tangent au champ à chacun de ses points
     \end{solution}
     \item Donner la définition de tube de champs \par
     \begin{solution}
      Ensemble des lignes de champ qui s'appuient sur un contour fermé 
     \end{solution}
\end{enumerate}
\end{document}
